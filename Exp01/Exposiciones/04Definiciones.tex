%%%%%%%%%%%%%%%%%%% Especificación:
\begin{frame}{Reloj Vectorial:}{Algoritmo.}
  \justifying
  \textbf{Una primera aproximación.} Inicialmente
  todos los procesos disponen de un vector con
  entradas igual al número total de procesos, este
  vector debe estar inicializado en $0$ para cada
  entrada. A continuación se describe el algoritmo:
  \begin{itemize}
  \item[$\blacktriangleright$] Si $p_i$ produce un evento, entonces:
    \begin{enumerate}
    \item[(1)] $Vc_i[i] \leftarrow Vc_i[i] + 1$;
    \item[(2)] Produce un evento $e$ caracterizado por $Vc_i[1, \dotsm, n]$.
    \end{enumerate}
  \item[$\blacktriangleright$] Cuando $p_i$ envia un mensaje a $p_j$, entonces:
    \begin{enumerate}
    \item[(3)] $Vc_i[i] \leftarrow Vc_i[i] + 1$;
    \item[(4)] \code{send(\textlangle msj, $Vc_i$[1, $\dotsm$, n] \textrangle)} a $p_j$.
    \end{enumerate}
  \item[$\blacktriangleright$] Cuando $p_j$ recibe un mensaje, entonces:
    \begin{enumerate}
    \item[(3)] $Vc_j[j] \leftarrow Vc_j[j] + 1$;
    \item[(4)] $Vc_j[1, \dotsm, n] \leftarrow \forall_{k \in [1, \dotsm, n]}
      \code{max}(Vc_i[k], Vc_j[k])$.
    \end{enumerate}
  \end{itemize}
\end{frame}
