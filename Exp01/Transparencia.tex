\documentclass[9pt]{beamer}

%~~~~~~~~~~~~~~~~~~~~~~~~~~~~~~~~~~~~~~~~~~~~~~~~~~~~~~~~~~~~~~~~~~~~~~~~~~~~~~
% Code
\newcommand{\code}[1]{\textcolor{white!25!black}{\texttt{#1}}}
%~~~~~~~~~~~~~~~~~~~~~~~~~~~~~~~~~~~~~~~~~~~~~~~~~~~~~~~~~~~~~~~~~~~~~~~~~~~~~~

%~~~~~~~~~~~~~~~~~~~~~~~~~~~~~~~~~~~~~~~~~~~~~~~~~~~~~~~~~~~~~~~~~~~~~~~~~~~~~~
% Include Figure
\usepackage{graphicx}
\usepackage{subcaption}
\usepackage{wrapfig}
%~~~~~~~~~~~~~~~~~~~~~~~~~~~~~~~~~~~~~~~~~~~~~~~~~~~~~~~~~~~~~~~~~~~~~~~~~~~~~~

%~~~~~~~~~~~~~~~~~~~~~~~~~~~~~~~~~~~~~~~~~~~~~~~~~~~~~~~~~~~~~~~~~~~~~~~~~~~~~~
% Use roboto Font (recommended)
\usepackage[sfdefault]{roboto}
\usepackage[utf8]{inputenc}
\usepackage[T1]{fontenc}
%~~~~~~~~~~~~~~~~~~~~~~~~~~~~~~~~~~~~~~~~~~~~~~~~~~~~~~~~~~~~~~~~~~~~~~~~~~~~~~

%~~~~~~~~~~~~~~~~~~~~~~~~~~~~~~~~~~~~~~~~~~~~~~~~~~~~~~~~~~~~~~~~~~~~~~~~~~~~~~
% Define where theme files are located. ('/styles')
\usepackage{styles/fluxmacros}
\usefolder{styles}
% Use Flux theme v0.1 beta
% Available style: asphalt, blue, red, green, gray
\usetheme[style=asphalt]{flux}
%~~~~~~~~~~~~~~~~~~~~~~~~~~~~~~~~~~~~~~~~~~~~~~~~~~~~~~~~~~~~~~~~~~~~~~~~~~~~~~

%~~~~~~~~~~~~~~~~~~~~~~~~~~~~~~~~~~~~~~~~~~~~~~~~~~~~~~~~~~~~~~~~~~~~~~~~~~~~~~
% Extra packages for the demo:
\usepackage{booktabs}
\usepackage{colortbl}
\usepackage{ragged2e}
\usepackage{schemabloc}
%~~~~~~~~~~~~~~~~~~~~~~~~~~~~~~~~~~~~~~~~~~~~~~~~~~~~~~~~~~~~~~~~~~~~~~~~~~~~~~
%~~~~~~~~~~~~~~~~~~~~~~~~~~~~~~~~~~~~~~~~~~~~~~~~~~~~~~~~~~~~~~~~~~~~~~~~~~~~~~
% Informations
\title{Análisis de Algoritmos II}
\subtitle{Un algoritmo de barrido de línea para agrupación espacial.\\
  \textbf{Profesores:}\\
  Jorge Urrutia Galicia\\
  Adriana Ramírez Vigueras\\
  Diego Jesús Favela Nava.}
\author{Aguilera Moreno Adrian.}
\institute{Facultad de Ciencias, UNAM}
\date{\today}
\titlegraphic{Imagenes/im1.png}
%~~~~~~~~~~~~~~~~~~~~~~~~~~~~~~~~~~~~~~~~~~~~~~~~~~~~~~~~~~~~~~~~~~~~~~~~~~~~~~

\begin{document}

% Generate title page
\titlepage

\begin{frame}
 \frametitle{Tabla de contenido.}
 \tableofcontents
\end{frame}
%\input{./Apartado01}
\section{Introducción.}
% Introducción
{\setbeamertemplate{background}{
\includegraphics[width=\the\paperwidth,height=\the\paperheight]{images/White.png}}
\begin{frame}
  \frametitle{Introducción}
  \framesubtitle{Problemas de visibilidad.} %%Subtítulo de la diapositiva (opcional)
  Determinar regiones de visibilidad de un objeto geométrico bajo restricciones es
  un problema muy estudiado en \textit{Geometría Computacional}.
  
  \centering \includegraphics[width=0.15 \paperwidth]{images/Visibility.png}
\end{frame}

\begin{frame}
  \frametitle{Introducción}
  \framesubtitle{Polígono de visibilidad.} %%Subtítulo de la diapositiva (opcional)
  Definimos un \underline{Polígono de visibilidad} como el polígono formado a partir
  de un punto $q$ dentro de un polígono dado, digamos $P$. Entonces, el polígono
  de visibilidad de $q$ definido como
  \[V(q) = \{p \in P\ \big|\ q \text{ visualiza a } p\}.\]
\end{frame}

\begin{frame}
  \centering \includegraphics[width=0.55 \paperwidth]{images/V(q)01.png}
\end{frame}

\begin{frame}
  \centering \includegraphics[width=0.55 \paperwidth]{images/V(q)02.png}
\end{frame}

\begin{frame}
  \centering \includegraphics[width=0.55 \paperwidth]{images/V(q)03.png}
\end{frame}

\begin{frame}
  \centering \includegraphics[width=0.55 \paperwidth]{images/V(q)04.png}
\end{frame}

\begin{frame}
  \centering \includegraphics[width=0.55 \paperwidth]{images/V(q)05.png}
\end{frame}

\begin{frame}
  \centering \includegraphics[width=0.55 \paperwidth]{images/V(q)06.png}
\end{frame}

\begin{frame}
  \centering \includegraphics[width=0.55 \paperwidth]{images/V(q)07.png}
\end{frame}

\begin{frame}
  \centering \includegraphics[width=0.55 \paperwidth]{images/V(q)08.png}
\end{frame}

\begin{frame}
  \centering \includegraphics[width=0.55 \paperwidth]{images/V(q)09.png}
\end{frame}

\begin{frame}
  \centering \includegraphics[width=0.55 \paperwidth]{images/V(q)10.png}
\end{frame}

\begin{frame}
  \centering \includegraphics[width=0.55 \paperwidth]{images/V(q)11.png}
\end{frame}

\begin{frame}
  \frametitle{Introducción}
  En partícular, podemos tener distintos tipos de polígonos.
  
  \begin{figure}
    \centering
    \includegraphics[width=0.75 \paperwidth]{./images/Casos.png}
    \caption*{Distintos casos para encontrar polígonos de visibilidad.}
  \end{figure}
  
  \textbf{Obs.} El polígono de visibilidad no, necesariamente, tiene que ser
  un polígono acotado.
\end{frame}


\subsection{Categorías.}
\input{./02CategoriasAgrupamientos}

\subsection{Alternativas.}
%%%%%%%%%%%%%%%%%%%%%%%%%%%%%%%% Introducción:
\begin{frame}[fragile]{Alternativas.}{}
  \begin{wrapfigure}{r}{0.25\textwidth} %this figure will be at the right
    \centering
    \includegraphics[width=0.25\textwidth]{./Imagenes/Aleatorios.png}
    \caption*{Aleatorios.}
  \end{wrapfigure}
  
  Algunas alternativas para agrupar son:
  \begin{enumerate}
  \item \textbf{Redes neuronales}.
  \item \textbf{k-means $+$ Algoritmos genéticos}.
  \item \textbf{Muestreos aleatorios}.
  \end{enumerate}

%  \begin{wrapfigure}{l}{0.25\textwidth}
%    \centering
%    \includegraphics[width=0.25\textwidth]{Imagenes/RedNeuronal.jpg}
%  \end{wrapfigure}
  \begin{justify}  
    Algunas alternativas para agrupar basadas en el entrenamiento inteligente,
    búsquedas aleatorias (como las heurísticas), y uso de algoritmos genéticos
    (como las colonias de hormigas) son recurridas cuando no podemos garantizar
    un ``buen'' agrupamiento.
  \end{justify}
  
  \begin{figure}
    \centering
    \begin{subfigure}[b]{0.3\textwidth}
      \includegraphics[width=\textwidth]{./Imagenes/RedNeuronal.jpg}
      \caption*{Redes Neuronales.}
    \end{subfigure}
    \begin{subfigure}[b]{0.3\textwidth}
      \includegraphics[width=\textwidth]{./Imagenes/Geneticos.jpeg}
      \caption*{K-means + Genéticos.}
    \end{subfigure}
  \end{figure}
\end{frame}


\subsection{Agrupación Espacial.}
%%%%%%%%%%%%%%%%%%%%%%%%%%%%%%%% Introducción:

\begin{frame}[fragile]{Agrupación Espacial:}{Propuestas I.}
  La agrupación espacial es un subconjunto espacial de agrupación. Este tipo
  de agrupamiento es relacionado, con frecuencia, a métodos gráficos.
  \begin{figure}
    \centering
    \begin{subfigure}[b]{0.3\textwidth}
      \includegraphics[width=\textwidth]{./Imagenes/LeyGeo.png}
      %\caption*{Redes Neuronales.}
    \end{subfigure}
    \begin{subfigure}[b]{0.2\textwidth}
      \includegraphics[width=\textwidth]{./Imagenes/Waldo_Tobler_2007.jpg}
      %\caption*{K-means + Genéticos.}
    \end{subfigure}
    \caption*{1era ley de la geografía.}
  \end{figure}
  \textbf{Propuestas:}
  \begin{itemize}
  \item Zahn sugiere trabajar con un gráfico completo (con vértices cada elemento en el espacio),
    construir el árbol de expansión mínima y eliminar los ``bordes'' más largos comparando las
    longitudes de los arcos con la longitud promedio, eliminando aquellos con longitud mayor al
    doble de la longitud promedio.
  \end{itemize}
\end{frame}

%%%%%%%%%%%%%%%%%%%%%%%%%%%%%%%% Introducción:

\begin{frame}[fragile]{Agrupación Espacial:}{Propuestas II.}
  \textbf{Propuestas:}
  \begin{itemize}
  \item Narendra sugiere el uso de diagramas de Voronoi para agrupar en
    tiempo $\mathcal{O}(n \log n)$. El problema de esta solución es que
    los algoritmos son dificiles de implementar.
  \item Kang usó triangulaciones de Delaunay y un diagrama dual de Voronoi.
    Después de construir la triangulación en $\mathcal{O}(n \log n)$, eliminamos
    las aristas con longitudes mayores a $d$.
  \item Yujian presentó un algoritmo de agrupamiento en subárboles máximos en
    distancia.
  \end{itemize}
  \textbf{¿Problemas? ...}
\end{frame}


%\def\beamer@mytheme@style{green}

%\subsection{Algoritmo.}
%\input{./04Definiciones} % Algoritmo.
%\input{./05Definiciones} % Propagación del tiempo vectorial.
%\input{./06Definiciones} % Propiedades.
%\input{./07Definiciones} % Reducción de costo en la comparación de dos vectores.
%\input{./08Definiciones} % Relación del tiempo vectorial y estados globales.

%\subsection{Desventajas.}
%\input{./09Desventajas}  % Desventajas.

% The [plain] causes the headlines, footlines, and sidebars
% to be suppressed. Useful for showing large pictures

% TODO. Sin implementar.
%\input{./Apartado02}
%\section{Aplicaciones.}
%\subsection{El caso DynamoDB.}
%\input{./Dynamo}
%\input{./Dynamo1}
%\input{./Dynamo2}
%\input{./Dynamo3}
%\input{./Dynamo4}
%\input{./Dynamo5}
%\input{./Dynamo6}
%\input{./Dynamo7}
%\input{./Dynamo8}
%\input{./Dynamo9}
%\input{./Dynamo10}
%\input{./Dynamo11}
%%%%%%%%%%%%%%%%% Nat:
% Solo es el orden, cambia lo que quieras (títulos). Perdón.
%----------------------------------------------------------
%\subsection{Relojes vectoriales dinámicos.}   %%%%%%%%%%%%%%%%%%%% Nat, aquí.  %%%%%%%%%%%%%%%%%%%%
%\input{./RVDinamicos01}
%\input{./RVDinamicos02}
%\input{./RVDinamicos03}
%\subsection{Seguimiento del predecesor inmediato (IPT).}
%\input{./IPT01}
%\input{./IPT02}
%\input{./IPT03}
%\input{./IPT04}
%\input{./IPT05}
%----------------------------------------------------------
%\subsection{Detección de una conjunción de predicados locales estables.}
%\input{./ConjunctionSatableLocalPredicates01}
%\input{./ConjunctionSatableLocalPredicates02}
%\subsection{Un problema de conjuntos: Conjuntos Posibles e Imposibles.}
%\input{./ProblemaConjuntos01}

%\input{./Apartado03}
%%%%%%%%%%%%%% Bloom: extra.
%\section{Relojes de Bloom.}
%\subsection{Filtro Bloom.}
%\input{./RelojesBloom01}
%\input{./RelojesBloom02}
%\input{./RelojesBloom03}
%\subsection{Algoritmo.}
%\input{./RelojesBloom04}
%\input{./RelojesBloom05}
%\input{./Agradecimientos}
\end{document}
