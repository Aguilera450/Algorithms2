%%%%%%%%%%% Exposición 02
\subsection[Expositor: Raúl Nuño Valdés.]{Un algoritmo de barrido de línea y su aplicación en espirales interiores (pocketing).}
\textbf{Obejetivo:} Construcción de espirales internas a un polígono dado. \newline

\textbf{Desarrollo:} Una alternaiva mencionada es usar los diagramas de Voronoi. Sin embargo, la
estrategia seguida es:
\begin{enumerate}
\item Ordenar nuestro arreglo de puntos.
\item Crear nuestro polígono. \newline
  
  En este punto nos interesa encontrar polígonos válidos, esto lo logramos búscando en
  el orden de las manecillas del reloj: empezamos en un punto y recorremos hasta toparnos
  con que no podemos avanzar y eliminamos esa arista (pues no será parte de nuestro polígono
  admisible).
\item Eliminar colineales.\newline

  No tiene sentido tener puntos colineales en el polígono, pues construiremos
  las espirales internas a partir de estos (todos) y solo nos interesa tener segmentos, sin importar si
  hay más de dos puntos en ellos.
\item Encontrar un extremo (caso partícular del extremo izquierdo).\newline
  
  Esto lo usamos para empezar el barrido y lo podemos encontrar en tiempo constante si ordenamos nuestro
  arreglo o en tiempo líneal en otro caso.
\item Barrido de línea. \newline
  
  Nos toma $\mathcal{O}(\log n)$ por punto, pues empezamos a contruir las espirales.
\item Encontrar intersecciones entre monótonas.\newline
  
  Las intersecciones serán parte de la construcción de la espiral.  
\item Obtener el arreglo de intersecciones. \newline
  
  Lo obtenemos del paso anterior en $\mathcal{O}(n \log n)$.
\end{enumerate}

\textbf{Obs.} El algoritmos consiste en ir acotando la súperficie interna del polígono e ir formando las espirales en estas
áreas. Debemos saber cuándo parar, esto es cuándo acotamos la sección llamada isla, en este punto el polígno debe tener
espirales internas y tener estas construcciones terminadas.\newline

\textit{Análisis de complejidad.} La complejidad esta contenida en
\[\therefore\: \mathcal{O}(n) + \mathcal{O}(n \log n) = \mathcal{O}(n \log n).\]
