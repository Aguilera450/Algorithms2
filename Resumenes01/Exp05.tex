%%%%%%%%%%% Exposición 03
\subsection[Expositor: Ares Gael Castro Romero.]{Triangulaciones de Graham y hamiltonianas con centro.}
\textbf{Obejetivo:} Triangular una nube de puntos de tal manera que podamos encontrar caminos hamiltonianos
que entren por una arista de un triángulo\footnote{No cualquier triángulo} y salgan por otra, finalmente
debemos encontrar un ciclo hamiltoniano.\newline

\textbf{Desarrollo:} Dividamos este tema en dos subtemas, estos son:
\begin{enumerate}
\item \textit{Caminos hamiltonianos en triangulaciones de Graham.} Definamos un \underline{Triángulo
  separador} $T_s$ como un triángulo del cuál sus aristas son parte de aristas en la triángulación y, además,
  existe al menos un punto contenido por $T_s$.\newline
  
  Al tener la triangulación $T$ de Graham sabemos que todo triángulo separador tiene como vértice a $v_0$ desde
  dónde se originó la triangulación. Entonces, encontrar la gráfica dual es sencillo desde este punto, pero
  ¿Qué es la gráfica dual? Es la gráfica que se forma a partir de los centros de triángulos generados en la
  triangulación y tales que son adyacentes si y sólo si existe una única arista entre ellos\footnote{Sólo son separados
  por una arista.}. La gráfica dual no es una trayectoria o ábol, entonces nos quedaremos con aristas a manera
  que cada arista sea de corte\footnote{Arista de corte. Sea $e$ una arista tal que al quitarla la gráfica se vuelve inconexa,
  entonces $e$ es de corte.} \newline
  
  El árbol generado anteriormente es un camino hamiltoniano tal que pasa por todos los triángulos generados por
  la triangulación de graham.
  
\item \textit{Caminos hamiltonianos en triangulaciones basadas en un centro.} A partir de un centro generamos triángulos
  que contengan una arista en el cierre convexo y que preserven la nube de puntos como una gráfica plana. Queremos buscar
  trayerctorias que pasen por todos los puntos en cada triángulo y que se conecte en forma de ciclo hamiltoniano. Las trayecotorias
  se conectan saliendo y entrando por los triángulos generados con el punto central y por tanto tenemos un ciclo hamiloniano.
\end{enumerate}
