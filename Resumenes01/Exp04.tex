%%%%%%%%%%% Exposición 03
\subsection[Expositor: Zayra Paulina Galván Ordoñez.]{Un problema simple de caminos alternos.}
\textbf{Obejetivo:} Resolver el problema de Putman para una cadena círcular partícular y dar el conjunto de posibles caminos
en una gráfica que se pueden tomar para llegar a la palabra dada, en caso de ser admisible. Esto es, ¿de cuántas maneras
podemos formar la cadena siempre que sea admisible?.\newline

\textbf{Desarrollo:} Nos basamos en las reglas de construcción:
\begin{itemize}
\item La palabra vacía es válida.
\item $abW$, dónde $W$ es no terminal (en adelante se omite la explicación).
\item $aWb$.
\item $bWa$.
\item $Wab$.
\end{itemize}
ahora, debemos contar las posiciones en la palabra círcular $c$. Enumeremos carácter
a carácter con $s_i$ y cuán queramos acceder a un carácter $j > |c|$, entonces buscaremos
el carácter en la posición $j \mod c$. Utilizaremos la notación $W(i, j)$ tal que iniciaremos
en el carácter $i$ y elijiremos los siguientes $k \cdot j$ carácteres de manera círcular y donde
$k$ es el número de carácteres de la palabra por nivel de jerárquia. \newline

¿Qué son los niveles de jerárquia? Les llamo partícularmente a los niveles que se forman en nuestra
gráfica, iniciando con la vacía, luego con $W(i,1), W(i,2), \dotsm$\newline

Ya definidos los niveles, entonces tenemos como ``fuente'' a la palabra vacía y como ``sumidero'' a la
palabra círcular objetivo, así basta bajar desde la fuente al sumidero en forma de aristas orientadas
para saber cuántas maneras hay de formar la palabra. Las palabras relacionadas por una arista serán
aquellas que procedan por medio de una regla de producción (de las ya mencionadas). \newline

Si la palabra círcular es no-admisible, entonces no hay manera de llegar por un camino desde la fuente
al sumidero. \newline

\textit{Análisis de complejidad.} Basta ver que en el pero de los casos es necesario construir la gráfica
que por cada nodo inicial (donde nace la arista orientada) tenga relación con todos los nodos en el siguiente
nivel, en general esto es un orden cuadrático y por tanto la complejidad de nuestro algoritmo es $\mathcal{O}(n^2)$.
