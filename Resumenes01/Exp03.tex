%%%%%%%%%%% Exposición 03
\subsection[Expositor: Juan Manuel Díaz Quiñonez.]{Localización de puntos planos mediante árboles de búsqueda persistentes.}
\textbf{Obejetivo:} Localización de puntos.\newline

\textbf{Desarrollo:} Segementamos nuestro espacio de búsqueda, por medio de un barrido de línea, en franjas.
Ahora necesitamos guardar la información de una manera no costosa y luego realizar una búsuqeda en estas
franjas. Nosotros emplearemos un árbol rojo-negro y por tanto bastará descender por el árbol para encontrar
el espacio dónde se localiza nuestro punto. Para poder tener nuestras frangas almacenadas de manera eficiente
haremos persistente nuestro árbol y con ``persistente'' nos referimos a que queremos que el árbol guarde de
alguna manera los estados del mismo. ¿Cómos hacemos lo anterios? Fusionaremos dos técnicas, estas son
\begin{enumerate}
\item Técnica 1. Realizar una copia de nodo por cada eliminación o reordenamiento, con sus respectivas referencias.
\item Técnica 2. Marcar referencias ``vivas'' (aquellas intactas) y ``muertas'' (aquellas cuáles nodos han sido borrado,
  al menos uno de ellos).
\end{enumerate}
con la fusión de ambas técnicas podemos garantizar que tendremos los estados por los que pasa el árbol durante su
construcción, sin necesidad de guardar estado por estado (este caso nos llevaría a un orden cuadrático). \newline

\textit{Análisis de complejidad.} En general, esto nos toma $\mathcal{O}(k + \log n)$ por nodo,
donde $k$ es la cantidad de punteros a ligar. Este valor es amortizado. Así, mantener la persistencia
nos toma
\[\therefore\: \mathcal{O}(n\cdot k + n \log n).\]
\textbf{Obs.} La complejidad es solo la de construir nuestro árbol.
