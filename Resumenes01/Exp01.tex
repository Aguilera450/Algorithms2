%%%%%%%%%%% Exposición 01
\subsection[Expositor: Adrián Aguilera.]{Un algoritmo de barrido de línea para agrupamiento espacial.}
\textbf{Obejetivo:} Agrupar un conjunto de puntos $P$ en el plano de manera
jerárquica, por medio de la distancia que los separa entre sí. \newline

\textbf{Desarrollo:} El algoritmo consiste en, dadas dos líneas
$s_1$ y $s_2$, y $P$ ordenado. Entonces barremos $P$ con ambas
líneas y dejando una separación $d$ entre $s_1$ y $s_2$. Diremos que
el espacio entre $s_1$ y $s_2$ será llamado ``manga''.\newline

En la primer iteración, agrupamos los puntos a una cercanía $\frac{d}{2}$ y coloreamos
bajo un mismo color los puntos que han sido agrupados en un mismo punto. En esta iteración
creamos el FRENTE DE AVANCE (AF), que esta dado por la trayectoria entre los puntos
coloreados a este punto. \newline

En la $i$-ésima iteración, sacamos los puntos de AF que no se encuentren dentro de la manga
y reconstruimos AF con los puntos dentro de la manga en este momento. Obtengamos las proyecciones, a AF,
de los puntos que vayan entrando a la manga, digamos $P(v)$. Así, asignamos un grupo a los puntos
tales que su proyección esté más cerca, por a lo más $\frac{d}{2}$,  a un punto con color asignado en AF.
Si la proyección no es cercana a nimgún punto en AF y el propio punto no es cercano a ningún punto con color,
entonces creamos un nuevo grupo. Si en algún momento dos grupos se ven unidos por un nuevo punto (que esta
cerca de puntos con colores distintos), entonces le asignamos color al grupo con menor cantidad de elementos.

\textbf{Obs.} Nótese que $P(v)$ podría no tener imagen cuando no existe una proyección en vértical con AF.
\newline

\textit{Análisis de complejidad.}
\begin{enumerate}
\item Ordenar $P$ nos toma $\mathcal{O}(n \log_2 n)$.
\item Mover la manga y realizar las iteraciones nos toma $\mathcal{O}(n \log n)$. Esto gracias a que $d$ es constante
  respecto a la altura (Y) del conjunto de puntos.
\item Recolorear grupos nos toma $c$, y a lo más $\mathcal{O}(c\cdot n)$.
\end{enumerate}
Concluimos una complejidad de
\[\therefore\: \mathcal{O}(n \log_2 n) + \mathcal{O}(n \log n) + \mathcal{O}(c\cdot n) = \mathcal{O}(n \log n).\]
