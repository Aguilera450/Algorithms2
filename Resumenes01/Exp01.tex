%%%%%%%%%%% Exposición 01
\subsection[Expositor: Adrián Aguilera.]{Un algoritmo de barrido de línea para agrupamiento espacial.}
\textbf{Obejetivo:} Agrupar un conjunto de puntos $P$ en el plano de manera
jerárquica, por medio de la distancia que los separa entre sí. \newline

\textbf{Desarrollo:} El algoritmo consiste en, dadas dos líneas
$s_1$ y $s_2$, y $P$ ordenado. Entonces barremos $P$ con ambas
líneas y dejando una separación $d$ entre $s_1$ y $s_2$. \newline

En la primer iteración, agrupamos los puntos a una cercanía $\frac{d}{2}$ y coloreamos
bajo un mismo color los puntos que han sido agrupados en un mismo punto. En esta iteración
creamos el FRENTE DE AVANCE (AF), que esta dado por la trayectoria entre los puntos
coloreados a este punto. \newline

En la $i$-ésima iteración,

